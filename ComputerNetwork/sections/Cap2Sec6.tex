\documentclass[../ComputerNetworks.tex]{subfiles}

\usepackage{hyperref}
\hypersetup{
  colorlinks=true,
  citecolor=cyan,
  linkcolor=magenta,
  urlcolor=blue}

\begin{document}

\subsection{Introduzione}

Tralasciando il costo di dei cavi, il costo di un opera pubblica per far passare un cavo è uguale al costo di far passare più cavi.

\subsection{Struttura del sistema telefonico}

Quando Bell brevettò il telefono, era già a conoscenza dei limiti imposti dalla grande distribuzione.
Introdusse così una gerarchia a due livelli di switching toll (uffici interurbani).

\subsection{The local Loop: Modems, ADSL, and Fiber}

Con "Local Loop"\footnote{Anche conosciuto come "last mile" (ultimo miglio).} si intende come è strutturato l'infrastruttura dalla cabina dell'ISP fino al DSL-Client.
\'E la connessione fisica o circuito che connette dai public switched telephone network alle attrezzature adibite alla connessione degli usofruitori del servizio di dati.

Il problema consiste che, a prescindere della bandwidth che ha a disposizione la DSLAM, la linea tra la centralina e la residenza del cliente è, nella maggior parte dei casi, la stessa linea telefonica che è limita in frequenza.
Tecnologie come \emph{ADSL}, riutilizzano la linea telefonica già presente.
Però la linea telefonica è stata progettata per trasmettere la voce umana per il telefono in analogico e quindi ha caratteristiche che non sono adatte alla trasmissione di dati digitali a banda elevata.

Sia i modem che le \emph{ADSL} devono tener conto delle limitazioni dei vecchi impianti telefonici.

Di recente si stanno adoperando misure per installare fibre ottiche adatte alla trasmissione a banda larga.

\subsection{Telephone Modems}

Per inviare bits in un generico canale fisico, devono essere convertiti in segnali analogici che possono essere trasmessi attraverso il canale, attraverso delle tecniche di modulazione digitale.
A destinazione, questi segnali trasmessi saranno poi convertiti nuovamente da analogici a bits.

Il dispositivo che esegue la conversione dei flussi di dati in analogico e in digitale, è chiamato \textbf{modem} che sta per "\emph{mo}dulator \emph{dem}odulator".

I modem telefonici sono utilizzati per mandare bits tra due computer collegati tra di loro attraverso una linea telefonica adibita alla trasmissione della voce.

La linea telefonica è limitata a trasmettere a frequenza di 3100Hz, sufficienti a trasmettere una conversazione.
Per contestualizzare, la bandwidth dello standard Ethernete o 802.11 è quattro ordini di grandezza superiore.

Per il teorema di Nyquist, perfino per una linea perfetta a 3000Hz, non è possibile trasmettere a più di 6000 baud.
In pratica, la maggior parte dei modem, manda ad una frequnza di 2400 baud, trasmettendo bits multipli per simbolo, permettendo traffico comunicazione Full Duplex\footnote{usando frequenze differenti per direzioni differenti}.

Questi modem utilizzano 0 volts per trasmettere segnale logico 0 e 1 volt per segnale logico 1, con 1 bit per simbolo.
Un avanzamento è usare quattro simboli differenti come attraversi QPSK, cosicché si abbia 2 bit/symbol e raggiungere una frequenza di trasmissione di 4800 bps.

Frequenze di trasmissione maggiori necessitano insiemi di simboli\footnote{in gergo \textbf{costellation}} più ampi; ciò però significa anche che perfino una piccola quantità di rumore di fondo nel rilevamento di ampiezza o fase può produrre errori.

Prima dell'avvento delle connesioni 56k, il limite di bandwidth era di 35-kbps dovuto alla presenza di local loop alle estremità dell'infrastruttura di trasmissione, uno dalla parte degli end office (uffici di distribuzione locali dei servizi) e i Digital Subscriber.

L'approccio intrapreso per le connessioni 56-kbps, consiste nel rimuovere il local loop, tipicamente quello che lega ISP e l'end office più vicino, e sostituirlo con una linea di alta qualità e bandwidth.

Le linee telefoniche sono state progettate per trasmettere in analogico la voce umana

Ogni canale telefonico ha 4000Hz\footnote{per maggiori informazioni si guardi \href{https://en.wikipedia.org/wiki/Voice_frequency}{Voice Frequency}} di capacità, includendo le guard bands.
Per poter ricostruire il segnale trasmesso attraverso la linea, sono necessari 8000 campioni al secondo, per questa data frequenza\footnote{per il teorema di Nyquist}.
\'E stato scelto, come standard,  il numero di 8 bit al secondo per campione di cui uno per ragioni di controllo; ciò permette il trasferimento di 56000 bit/sec.

\subsection{Digital Subscriber Lines}

Il motivo per cui i modem appena discussi erano lenti, è dovuto al limite tecnico della linea per il quale è stato ottimizzata tutto il sistema, ovvero per trasmettere la voce umana.

Nel punto in cui ogni local loop termina in un end office, il cavo è collegato ad un filtro che attenua ogni frequenza sotto i 300Hz e sopra i 3400Hz\footnote{il cutoff non è netto per cui si considera 4000Hz}.

Nel caso di una \emph{xDSL}\footnote{nome generico utilizzato per un generico Digital Subscriber LIne} il customer è collegato in un altro tipo di switch che non ha il filtro integrato, permettendo di utilizzare a pieno la capacità del local loop.
Il fattore limitante rimane ora il local loop che supporta circa 1MHz.





\biblio
\end{document}
